\documentclass[aspectratio=169]{beamer}

%\usepackage[brazil]{babel} % Set the environment to pt-br

%Theme options (allow multiple options):

% Research Group (researchgroup) options: gres (default), cbio, or both
% How to use: \usetheme[researchgroup=cbio]{CbioGres}

% Color theme (colortheme): dark, blue, lightblue, green
% How to use: \usetheme[colortheme=blue]{CbioGres}

\usetheme[colortheme=lightblue,researchgroup=both]{CbioGres}

\usepackage{blindtext}
\usepackage{moresize}

\usepackage[ruled,onelanguage]{algorithm2e}
\usepackage[backend=bibtex,style=numeric]{biblatex}
\hypersetup{pdfpagemode=FullScreen}
\bibliography{refs}

\usepackage{pgfpages}
%\setbeameroption{hide notes} % Only slides
%\setbeameroption{show only notes} % Only notes
\setbeameroption{show notes on second screen=right} % Both
\setbeamertemplate{note page}{\pagecolor{yellow!5}\vfill\insertnote\vfill}
\usepackage{palatino}

\title{What is the Vocabulary of Flaky Tests? \\An Extended Replication}

\author{Bruno Henrique Pachulski Camara$^{1,2}$,
		Marco Aurélio Graciotto Silva$^{3}$,
		Andre T. Endo$^{4}$,
		Silvia Regina Vergilio$^{2}$}
\institute{
    $^{1}$\textit{Centro Universitário Integrado}, Campo Mourão, PR, Brazil \\
    $^{2}$\textit{Department of Computer Science}, \textit{Federal University of Paraná}, Curitiba, PR, Brazil \\
    $^{3}$\textit{Department of Computing}, \textit{Federal University of Technology - Paraná}, \\Campo Mourão, PR, Brazil \\ 
    $^{4}$\textit{Department of Computing}, \textit{Federal University of Technology - Paraná}, \\Cornélio Procópio, PR, Brazil\\ 
}

\date{\today}

\subtitle{\tiny{ICPC, May 18--21, 2021, Madrid, Spain}}

\setcounter{showSlideNumbers}{1}

\begin{document}

	% 30's
	\frame{
		\titlepage

		\note[item] {Hello, my name is Bruno. I'm from Universidade Federal do Paraná and Centro Universitário Integrado from Brazil. And gonna present a paper "What is the Vocabulary of Flaky Tests? An Extended Replication".}
		\note[item] {This is a work guided by my advisors, Marco Graciotto, André Endo and Silvia Vergilho. }
	
	}

	
	\startprogressbar

	% 55's
	\begin{frame}		
		\frametitle{\large Introduction}

		\begin{enumerate}
			\item Flaky tests \\ \textcolor{ExecusharesGrey}{\footnotesize\hspace{1em} Tests who pass or fail when executed in the same software version}
			\item Is it possible to compile the vocabulary of the flaky tests?  \\ \textcolor{ExecusharesGrey}{\footnotesize\hspace{1em} What is the Vocabulary of Flaky Tests? \cite{Pinto-etal:2020}}
			\item Evaluation of Intra- and Inter-Project prediction \\ \textcolor{ExecusharesGrey}{\footnotesize\hspace{1em} Can we predict flaky tests with the tained classifiers?}
		\end{enumerate}

		\note[item] {Flaky tests are a recurring issue hurting the adoption of automated tests. A test case that may pass or fail non-deterministically in the same production code.}
		\note[item] {So, a promising approach is to collect static data of automated tests and use them to predict their flakiness. Is possible to use test code as a feature? Pinto et Al. proposed it and our first goal is to replicate the training and prediction with the same dataset but with the different Machine Learning platforms and algorithms. }
		\note[item] {Then, we propose an extension to validate the trained model against another dataset to validate the effectiveness of the model on identify flaky tests Intra- and Inter-projects. }

	\end{frame}	
	
	%Sections
	\section{Replication}
	
		\begin{frame}

			\frametitle{\large RQ$_1$ -- Can we obtain similar results by using other ML algorithms or the same algorithms with distinct implementations?}

			\begin{enumerate}

				\item 

			\end{enumerate}

			\note[item] {testes}

		\end{frame}

		\begin{frame}

			\frametitle{\large RQ$_{1.1}$ -- How accurately can we predict test flakiness based on source code identifiers in the test cases?}

			\begin{enumerate}

				\item 
				
			\end{enumerate}

			\note[item] {testes}

		\end{frame}

		\begin{frame}

			\frametitle{\large RQ$_{1.2}$ -- What value of different features add to the classifier?}

			\begin{enumerate}

				\item 
				
			\end{enumerate}

			\note[item] {testes}

		\end{frame}

		\begin{frame}

			\frametitle{\large RQ$_{1.3}$ -- Which test code identifiers are most strongly associated with test flakiness?}

			\begin{enumerate}

				\item 
				
			\end{enumerate}

			\note[item] {testes}

		\end{frame}

	\section{Extension}
	
		\begin{frame}

			\frametitle{\large RQ$_2$ -- Are the results valid for other datasets including different projects?}

			\begin{enumerate}

				\item 

			\end{enumerate}

			\note[item] {testes}

		\end{frame}

		\begin{frame}

			\frametitle{\large RQ$_{2.1}$ -- Can a trained classifier be successfully applied within the same projects (i.e., intra-project)?}

			\begin{enumerate}

				\item 

			\end{enumerate}

			\note[item] {testes}

		\end{frame}

		\begin{frame}

			\frametitle{\large RQ$_{2.2}$ -- Can a trained classifier be successfully applied to other projects (i.e., inter-projects)?}

			\begin{enumerate}

				\item 

			\end{enumerate}

			\note[item] {testes}

		\end{frame}


		
	\section{Conclusions}

		\begin{frame}
			\frametitle{\large Closing Thoughts}
			

			\note[item] {testes}

		\end{frame}
	
		\frame{\acknowledgmentpage}

\end{document}